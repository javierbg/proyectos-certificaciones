\section{PMI}
\begin{frame}{PMI}
	PMI (\foreign{Project Management Institute})
	\begin{itemize}
		\item Organización internatcional (sede en Pensilvania, EE.UU.).
		\item Asocia a profesionales de la gestión de proyectos.
		\item Presente en casi 100 países, organizado por ``capítulos''
		\begin{itemize}
			\item En España: Madrid, Barcelona y Valencia.
		\end{itemize}
		\item Edita el PMBOK (\foreign{Project Management Body of Knowledge}).
	\end{itemize}
\end{frame}

\subsection{PMP}
\begin{frame}[allowframebreaks]{PMP}
	PMP (\foreign{Project Management Professional})
	\begin{itemize}
		\item Una de las certificaciones \textbf{más reconocidas} en el ámbito de gestión de proyectos.
	\end{itemize}
	
	\begin{block}{Requisitos}
		\begin{itemize}
			\item 35 horas de formación específica en gestión de proyectos.
			\item Número mínimo de horas demostrables dedicadas a la gestión de proyectos.
			\begin{itemize}
				\item 4500 horas en 3 años si se tienen estudios universitarios.
				\item 7500 horas en 5 años si sólo se tienen estudios secundarios.
			\end{itemize}
		\end{itemize}
	\end{block}
	
	\begin{block}{Examen}
		Cumpliendo los anteriores requisitos puedes acceder a un examen que abarca todas las fases de la gestión de proyectos:
		\begin{enumerate}
			\item Inicio
			\item Planificación
			\item Ejecución
			\item Seguimiento y control
			\item Cierre
		\end{enumerate}
	\end{block}
	
	\begin{itemize}
		\item El certificado \textbf{no es vitalicio}. Es necesario \textbf{renovarlo cada 3 años}.
		
		\item \textbf{Bastante demanda}: 112 resultados distintos de ofertas de trabajo en Infojobs (octubre de 2016).
	\end{itemize}
\end{frame}