\section{PMI}
\begin{frame}{PMI}
	PMI (\foreign{Project Management Institute})
	\begin{itemize}
		\item Organización internacional (sede en Pensilvania, EE.UU.).
		\item Asocia a profesionales de la gestión de proyectos.
		\item Presente en casi 100 países, organizado por ``capítulos''
		\begin{itemize}
			\item En España: Madrid, Barcelona y Valencia.
		\end{itemize}
		\item Edita el PMBOK (\foreign{Project Management Body of Knowledge}).
	\end{itemize}
\end{frame}

\subsection{PMP}
\begin{frame}[allowframebreaks]{PMP}
	PMP (\foreign{Project Management Professional})
	\begin{itemize}
		\item Una de las certificaciones \emph{más reconocidas} en el ámbito de gestión de proyectos.
	\end{itemize}
	
	\begin{block}{Requisitos}
		\begin{itemize}
			\item 35 horas de formación específica en gestión de proyectos.
			\item Número mínimo de horas demostrables dedicadas a la gestión de proyectos.
			\begin{itemize}
				\item 4500 horas en 3 años si se tienen estudios universitarios.
				\item 7500 horas en 5 años si sólo se tienen estudios de bachillerato.
			\end{itemize}
		\end{itemize}
	\end{block}
	
	\begin{block}{Examen}
		Cumpliendo los anteriores requisitos puedes acceder a un examen que abarca todas las fases de la gestión de proyectos:
		\begin{enumerate}
			\item Inicio
			\item Planificación
			\item Ejecución
			\item Seguimiento y control
			\item Cierre
		\end{enumerate}
		\begin{itemize}
			\item 200 preguntas de opción múltiple
			\item 4 horas para completarlo
		\end{itemize}
	\end{block}
	
	\framebreak
	
	\begin{itemize}
		\item El certificado \emph{no es vitalicio}. Es necesario \emph{renovarlo cada 3 años} obteniendo \emph{PDUs}.
		
		\item \emph{Bastante demanda}: 112 resultados distintos de ofertas de trabajo en Infojobs (octubre de 2016).
	\end{itemize}
\end{frame}

\subsection{CAPM}

\begin{frame}[allowframebreaks]{CAPM}
	CAPM (\foreign{Certified Associate in Project Management})
	\begin{itemize}
		\item Certificación similar a PMP apropiada para casos donde no se tiene la experiencia necesaria para ésta.
		
		\item Enfocado a la gestión en general, no tanto a la dirección.
	\end{itemize}
	
	\begin{block}{Requisitos}
		\begin{itemize}
			\item Estudios mínimos de bachillerato.
			\item Dos opciones:
			\begin{itemize}
				\item 1500 horas demostrables dedicadas a la gestión de proyectos.
				\item 23 horas de formación específica en gestión de proyectos.
			\end{itemize}
		\end{itemize}
	\end{block}
	
	\framebreak
	
	\begin{block}{Examen}
		Cumpliendo los anteriores requisitos puedes acceder a un examen que cubre todos los el material del PMBOK.
		\begin{itemize}
		\item 150 preguntas de opción múltiple
		\item 3 horas para completarlo
		\end{itemize}
	\end{block}
	
	\begin{itemize}
		\item El certificado \emph{no es vitalicio}. Es necesario \emph{renovarlo cada 5 años}.
		
		\item \emph{Baja demanda}: Suele demandarse PMP y, en algunos casos, opcionalmente CAPM en su lugar.
	\end{itemize}
\end{frame}