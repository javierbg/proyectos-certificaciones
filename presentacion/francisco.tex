\section{Certificaciones SCRUM y SCRUM MASTER}
\begin{frame}{SCRUM}
	Scrum y Scrum Master
	\begin{itemize}
		\item Scrum es un marco de trabajo ágil para desarrollar y mantener productos complejos.
		\item Scrum Master es la persona conocedora del proceso SCRUM.
	\end{itemize}

	\framebreak

	Las dos principales organizaciones certificadoras son:
	\begin{itemize}
		\item Scrum Alliance (Certificación CSM)
		\item Scrum.org (Certificación PSM)
	\end{itemize}

\end{frame}

\subsection{Certificación CSM}

\begin{frame}[allowframebreaks]{Certificación CSM}
	Certificación CSM \foreign{Certified Scrum Master}
	\begin{itemize}
		\item Curso obligatorio de 16 horas.
		\item Examen obligarorio de 35 preguntas verdadero/falso y de múltiple opción.
		\item Renovaciones obligatorias cada dos años.
	\end{itemize}

\end{frame}

\subsection{Certificación PSM}

\begin{frame}[allowframebreaks]{Certificación PSM}
	Certificación PSM \foreign{Proffesional Scrum Master}

	\begin{itemize}
		\item Consta de tres niveles de certificacion PSM I, PSM II y PSM III
		\item No requiere la realización de ningún curso.
		\item No requiere de renovaciones.
	\end{itemize}

	\framebreak

	Certificación PSM I
	\begin{itemize}
		\item Conocimiento básico sobre las funciones del Scrum Master.
		\item Evaluación basada en The Scrum Guide
		\item Examen de una hora con 80 preguntas de múltiple elección y verdadero/falso.
	\end{itemize}

	\framebreak

	Certificación PSM II
	\begin{itemize}
		\item Conocimiento avanzado sobre las funciones de un Scrum Master
		\item Examen de 90 minutos
		\item 30 preguntas de múltiple elección, múltiple respuesta o verdadero/falso
	\end{itemize}

	\framebreak

	Certificación PSM III
	\begin{itemize}
		\item Conocimiento distinguido sobre las tareas del Scrum Master.
		\item Examen de 2 horas con preguntas de múltiple elección y casos prácticos.
		\item Requiere haber superado las certificaciones PSM I y PSM II
	\end{itemize}


\end{frame}


\section{Introducción al PMBOK}

\begin{frame}[allowframebreaks]{Introducción PMBOK: Propósito}
	
	\begin{itemize}
		\item Provee un marco de referencia formal para desarrollar proyectos. 
		\item Proporciona a los jefes de proyecto una guía para avanzar en los procesos necesarios para obtener resultados y alcanzar los objetivos.
		\item Renovaciones obligatorias cada dos años.
	\end{itemize}

\end{frame}

