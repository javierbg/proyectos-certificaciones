
\section{Certificaciones de SCRUM MASTER}
\begin{frame}[allowframebreaks]{Certificaciones de SCRUM MASTER}
	
	Scrum y Scrum Master
	
	\begin{itemize}
		\item Scrum es un marco de trabajo ágil para desarrollar y mantener productos complejos.
		\item Scrum Master es la persona conocedora del proceso SCRUM.
	\end{itemize}


	Las dos principales organizaciones certificadoras son:
	\begin{itemize}
		\item Scrum Alliance (Certificación CSM)
		\item Scrum.org (Certificación PSM)
	\end{itemize}

\end{frame}

\subsection{Certificación CSM}

\begin{frame}{Certificación CSM}
	Certificación CSM \foreign{Certified Scrum Master}
	\begin{itemize}
		\item Curso obligatorio de 16 horas.
		\item Examen obligarorio de 35 preguntas verdadero/falso y de múltiple opción.
		\item Renovaciones obligatorias cada dos años.
	\end{itemize}

\end{frame}

\subsection{Certificación PSM}

\begin{frame}[allowframebreaks]{Certificación PSM}
	Certificación PSM \foreign{Proffesional Scrum Master}

	\begin{itemize}
		\item Consta de tres niveles de certificacion PSM I, PSM II y PSM III
		\item No requiere la realización de ningún curso.
		\item No requiere de renovaciones.
	\end{itemize}

	\framebreak

	Certificación PSM I
	\begin{itemize}
		\item Conocimiento básico sobre las funciones del Scrum Master.
		\item Evaluación basada en The Scrum Guide
		\item Examen de una hora con 80 preguntas de múltiple elección y verdadero/falso.
	\end{itemize}

	\framebreak

	Certificación PSM II
	\begin{itemize}
		\item Conocimiento avanzado sobre las funciones de un Scrum Master
		\item Examen de 90 minutos
		\item 30 preguntas de múltiple elección, múltiple respuesta o verdadero/falso
	\end{itemize}

	\framebreak

	Certificación PSM III
	\begin{itemize}
		\item Conocimiento distinguido sobre las tareas del Scrum Master.
		\item Examen de 2 horas con preguntas de múltiple elección y casos prácticos.
		\item Requiere haber superado las certificaciones PSM I y PSM II
	\end{itemize}


\end{frame}


\section{La guía del PMBOK: Introducción.}

\subsection{Propósito}
\begin{frame}{Propósito}
	
	\begin{itemize}
		\item{Identifica un conjunto de buenas prácticas para la dirección de proyectos}
		\item{Proporciona un vocabulario común para los conceptos de la dirección de proyectos}
		\item{\emph{Es una guía, no una metodología}}
	\end{itemize}
\end{frame}

%%
% ahora explicas que los dos primeros capítulos de la guía PMBOK
% son una introducción a una serie de conceptos importantes en el 
% ámbito de la dirección de proyectos, y que hablaremos de algunos
% que hemos considerado más interesantes.
%%
\subsection{Portafolios, Programas y Proyectos}
\begin{frame}[allowframebreaks]{Portafolios, Programas y Proyectos}
	
	\begin{block}{Definición de \emph{Portafolios}}
		Un portafolio es un conjunto de proyectos, programas, subconjuntos de  portafolios y operaciones, que son gestionadas como un conjunto, \emph{con el fin de alcanzar los objetivos estratégicos de la empresa}.
	\end{block}

	% ejemplos de estrategias de empresa pueden ser por ejemplo 
	% "maximizar el rendimiento de las inversiones"

	\framebreak

	\begin{block}{Definición de \emph{Programa}}
		Un programa es un grupo de proyectos relacionados, suprogramas y actividades de programas, cuya gestión se realiza de forma coordinada para obtener beneficios que no se obtendrían si se gestionaran de forma individual.
	\end{block}

	% todo : buscar ejemplos de programa para explicarlo

	\framebreak
	\begin{center}
		\includegraphics[height=5cm]{figuras/proy_prog_port_00.png}

		Contexto de la dirección de proyectos
	\end{center}

\end{frame}


\subsection{El ciclo de vida del proyecto}
\begin{frame}{El ciclo de vida del proyecto}
	
	El ciclo de vida es la serie de fases por las que atraviesa un proyecto desde su inicio hasta su cierre, proporcionando un marco de referencia básico para dirigir el proyecto.

	\begin{center}
		\includegraphics[height=4.5cm]{figuras/ciclo_vida_00.png}

		Estructura genérica de ciclo de vida.
	\end{center}


	

\end{frame}

