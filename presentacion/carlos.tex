\section{IPMA}

\subsection{IPMA}
\begin{frame}[allowframebreaks]{IPMA}
	IPMA \foreign{International Project Management Association}
	\begin{itemize}
		\item Asociación Internacional dedicada al \textbf{ y promoción de la dirección de proyectos}.
		\begin{itemize}
			\item Scrum, Kanban, Lean, programación extrema (XP), desarrollo guiado por pruebas (TDD)...
		\end{itemize}
	\end{itemize}
	\begin{block}{Requisitos}
		\begin{itemize}
			\item 2000 horas de trabajo en equipo en proyectos en general
			\item 1500 horas de trabajo en equipo en proyectos ágiles
			\item 21 horas de formación presencial en metodologías ágiles
		\end{itemize}
	\end{block}
	\begin{block}{Examen}
		El temario se organiza en \textbf{dominios}
		\begin{enumerate}
			\item Principios y mentalidad ágiles
			\item Entrega orientada al valor
			\item Participación de las partes interesadas
			\item Rendimiento del equipo
			\item Planificación adaptativa
			\item Detección y resolución de problemas
			\item Mejora continua
		\end{enumerate}
		Estos temas abarcan el 80\% de la nota, el otro 20\% es abierto.
	\end{block}
	
	\framebreak
	
	\begin{itemize}
		\item Renovación (como miembro del \textbf{CCR}):
		\begin{itemize}
			\item Acumulación de \textbf{30 PDUs}
			\item Un examen a los 3 años
		\end{itemize}
		\item Demanda:
		\begin{itemize}
			\item Según PMI, es la certificación que más ha crecido.
			\item En Infojobs sólo 2 resultados (octubre 2016).
		\end{itemize}
	\end{itemize}
\end{frame}
