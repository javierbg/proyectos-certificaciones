\subsection{PMI-ACP}
\begin{frame}[allowframebreaks]{PMI-ACP}
	PMI-ACP \foreign{PMI Agile Certified Practitioner}
	\begin{itemize}
		\item Reconocer conocimiento en \textbf{principios y tecnologías ágiles}.
		\begin{itemize}
			\item Scrum, Kanban, Lean, programación extrema (XP), desarrollo guiado por pruebas (TDD)...
		\end{itemize}
	\end{itemize}
	\begin{block}{Requisitos}
		\begin{itemize}
			\item 2000 horas de trabajo en equipo en proyectos en general
			\item 1500 horas de trabajo en equipo en proyectos ágiles
			\item 21 horas de formación presencial en metodologías ágiles
		\end{itemize}
	\end{block}
	\begin{block}{Examen}
		El temario se organiza en \textbf{dominios}
		\begin{enumerate}
			\item Principios y mentalidad ágiles
			\item Entrega orientada al valor
			\item Participación de las partes interesadas
			\item Rendimiento del equipo
			\item Planificación adaptativa
			\item Detección y resolución de problemas
			\item Mejora continua
		\end{enumerate}
		Estos temas abarcan el 80\% de la nota, el otro 20\% es abierto.
	\end{block}
	
	\framebreak
	
	\begin{itemize}
		\item Renovación (como miembro del \textbf{CCR}):
		\begin{itemize}
			\item Acumulación de \textbf{30 PDUs}
			\item Un examen a los 3 años
		\end{itemize}
		\item Demanda:
		\begin{itemize}
			\item Según PMI, es la certificación que más ha crecido.
			\item En Infojobs sólo 2 resultados (octubre 2016).
		\end{itemize}
	\end{itemize}
\end{frame}

\section{PRINCE2}
\begin{frame}[allowframebreaks]{PRINCE2}
	PRINCE2 \foreign{PRojects IN Controlled Environments v2}
	\begin{itemize}
		\item Método basado en procesos para la gestión efectiva de proyectos de dominio público.
		\item Desde la organización se ofrecen distintas acreditaciones para certificar el conocimiento, a distintos de niveles, de la metodología.
	\end{itemize}
	
	\framebreak
	
	\begin{block}{Metodología}
		Se basa en 7 principios:
		\begin{itemize}
			\item Justificación comercial continua
			\item Aprender de la experiencia
			\item Roles y responsabilidades definidos
			\item Gestión por fases
			\item Gestión por excepción
			\item Orientación a productos
			\item Adaptación
		\end{itemize}
	\end{block}
	
	\framebreak
	
	\begin{block}{Acreditaciones}
		Las acreditaciones son incrementales.
		\begin{itemize}
			\item PRINCE2 Foundation: trabajar en un grupo que apoye siga la metodología PRINCE2.
			\item PRINCE2 Practitioner: aplicación y adaptación de PRINCE2 a una situación
			\begin{itemize}
				\item También accesible vía cualquiera de las acreditaciones del \textbf{PMI} o \textbf{IPMA}.
			\end{itemize}
			\item PRINCE2 Re-Registration: examen de renovación
			\begin{itemize}
				\item Debe pasarse cada 3 o 5 años
			\end{itemize}
			\item PRINCE2 Professional: gestionar PRINCE2 en todos los aspectos del ciclo de vida del proyecto.
			\begin{itemize}
				\item No hay examen escrito: trabajo en grupo en el estudio de un caso de proyecto ficticio
			\end{itemize}
		\end{itemize}
	\end{block}
\end{frame}