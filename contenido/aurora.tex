\subsection{PMI-ACP}

\emph{PMI Agile Cerfified Practioner (PMI-ACP)} es una certificación del \emph{Project Management Institute} certifica el conocimiento de los principios ágiles así como las habilidades con las tecnologías ágiles. Entre estas metodologías se encuentran \emph{Scrum, Kanban, Lean, programación extrema (XP) o el desarrollo guiado por pruebas (TDD)}.\\

Al ser una certificación exclusiva a PMI, sólo ellos la acreditan. La forma de acceder a esta acreditación es vía un examen on-line que ofrecen en su propia web. El precio varía entre los miembros del PMI, que es de 435,00\$, y los no miembros, que tienen que abonar 495,00\$. Por su parte, hacerse miembro de PMI no tiene requisitos y reporta beneficios como la inclusión en redes personales que se organizan en lo que llaman \emph{Local Chapters}, pero cuesta 129\$.\\

Para optar al examen, se requieren unos requisitos previos en forma de horas trabajo en relación a las metodologías ágiles:
\begin{itemize}
	\item 2 000 horas de experiencia general del proyecto trabajando en equipos. Un PMP actual o PgMP® va a satisfacer este requisito, pero no está obligado a solicitar el PMI-ACP.
	\item 1 500 horas de trabajo en equipos de proyectos ágiles o con metodologías ágiles. Esta disposición se suma a las 2 000 horas de experiencia general del proyecto.
	\item 21 horas presenciales de formación en prácticas ágiles.
\end{itemize}

Una vez se obtiene la certificación, el acreditado pasa a ser miembro del \emph{Programa de Certificación de Requirimientos (CCR)}. Como tal, tiene la obligación de participar en actividades de desarrollo profesional. La cantidad de actividades a realizar se regula mediante las llamadas \emph{Unidades de Desarrollo Profesional (PDU)}, que son una especie de créditos que se van obteniendo. \emph{PMI-ACP} establece que se necesita conseguir 30 PDUs en un plazo de 3 años para, entonces, optar a un examen de renovación de la certificación.\\

Esta certificación, al ser propiedad de PMI, cuenta con toda infraestructura de apoyo a nivel internacional del Instituto. Así, tiene validez en los 207 países en los que opera PMI, y a nivel local cuenta con los \emph{Local Chapters} ya mencionados en Madrid y Valencia.\\

Según PMI, es la certificación de su propiedad cuya demanda más ha crecido en los últimos años. Por esto, con una simple búsqueda en Internet se pueden encontrar multitud de cursos de preparación no oficiales a nivel nacional e internacional.


\section{PRINCE2}

PRojects IN Controlled Environments v2 (PRINCE2) es un método basado en procesos para la gestión efectiva de proyectos. Es de dominio público utilizado en una gran amplitud de campos, desde en gobiernos como el de Reino Unido hasta proyectos en el campo de las TIC.\\

Trata de cubrir, mediante lo que se conoce como \emph{Temáticas}, la Calidad, el Cambio, la estructura de roles del proyecto (Organización), los planes (Cuánto, Cómo, Cuando), el Riesgo y el Progreso del proyecto. El proyecto en cuestión debe ser justificado por un \emph{Business Case} (o estudio de viabilidad) que debe ser revisado durante el ciclo de vida del proyecto y justificarlo en todo momento como consecución de los beneficios esperados.\\

Se basa en 7 principios:
\begin{enumerate}
	\item Justificación comercial continua
	\begin{itemize}
		\item Se asegura de que hay un motivo justificable para iniciar el proyecto.
		\item La justificación se mantiene válida durante toda la vida del proyecto.
		\item Dicha justificación ha sido identificada, y aprobada.
	\end{itemize}
	\item Aprender de la experiencia
	\begin{itemize}
		\item Se recogen las experiencias anteriores, las que se van obteniendo a lo largo de la ejecución del proyecto, así como las lecciones aprendidas al cierre del mismo.
	\end{itemize}
	\item Roles y Responsabilidades definidos
	\begin{itemize}
		\item Asegurando que los intereses de los usuarios que van a usar el proyecto, los proveedores y el responsable del área de negocio están representados en la toma de decisiones.
	\end{itemize}
	\item Gestión por Fases
	\begin{itemize}
		\item Un proyecto que sigue la metodología PRINCE2 se planifica, se supervisa y se controla fase a fase.
	\end{itemize}
	\item Gestión por excepción
	\begin{itemize}
		\item Es decir, delegar la autoridad suficiente de un nivel de gestión al siguiente, dándole autonomía según unas tolerancias pautadas (de tiempo, coste, calidad, alcance, beneficio y/o riesgo) de manera que, de sobrepasar la tolerancia, se consulte al nivel superior como actuar.
	\end{itemize}
	\item Orientación a productos
	\begin{itemize}
		\item Centra la atención en la definición y entrega de productos, es decir, un proyecto no son un conjunto de tareas a realizar, sino que entrega productos (que se elaboran tras la ejecución de las tareas que sean necesarias).
	\end{itemize}
	\item Adaptación
	\begin{itemize}
		\item Asegurando que la metodología PRINCE2 y los controles a aplicar se basen en el tamaño, complejidad, importancia, capacidad y nivel de riesgo del proyecto.
	\end{itemize}	
\end{enumerate}


Como se ha dicho, aunque la metodología sea pública, desde la organización se ofrecen certificaciones para distintos dominios de dicha metodología. Estas certificaciones tienen validez en cualquier país y requieren de una renovación cada 3 años. Se pasan a listar a continuación:
\begin{itemize}
	\item \emph{PRINCE2 Foundation}: suficiente conocimiento y comprensión del método PRINCE2 para ser capaz de trabajar en un equipo de gestión de proyectos que usa este método. Precio: 455\geneuronarrow.
	\begin{itemize}
		\item Requisitos previos: ninguno
	\end{itemize}
	\item \emph{PRINCE2 Practitioner}: suficiente comprensión de cómo aplicar PRINCE2 en una situación escenario. El acreditado, con la dirección adecuada, será capaz de iniciar la aplicación del método a un proyecto real. Precio: 455\geneuronarrow.
	\begin{itemize}
		\item Requisitos previos: PRINCE2 Foundation o una de las certificaciones de PMI o IPMA (PRINCE2 Fundación, CAPM, PMP, IPMA Nivel D ®, IPMA Nivel C ®, IPMA Nivel B®, IPMA Nivel A®).
	\end{itemize}
	\item \emph{PRINCE2 Professional}: gestionar un proyecto PRINCE2 no complejo en todos los aspectos del ciclo de vida del proyecto.
	\begin{itemize}
		\item Requisitos previos: PRINCE2 Practitioner.
	\end{itemize}
	\item \emph{PRINCE2 Agile Practitioner}: aplicar los principios de gestión de proyectos de PRINCE2 y la combinación de conceptos ágiles como Scrum y Kanban.
	\begin{itemize}
		\item Requisitos previos: PRINCE2 Practitioner
	\end{itemize}
\end{itemize}