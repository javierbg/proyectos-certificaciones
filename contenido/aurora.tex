\subsection{PMI-ACP}

\foreign{PMI Agile Certified Practitioner} (PMI-ACP) es una certificación del \emph{Project Management Institute} que certifica el conocimiento de los principios ágiles así como las habilidades con las tecnologías ágiles. Entre estas metodologías se encuentran \emph{Scrum, Kanban, Lean, programación extrema (XP) o el desarrollo guiado por pruebas (TDD)}.\\

La forma de acceder a esta acreditación es vía un examen on-line que ofrecen en su propia web. El precio varía entre los miembros del PMI, que es de 435,00\$, y los no miembros, que tienen que abonar 495,00\$.\\

Para optar al examen, se requieren unos requisitos previos en forma de horas trabajo en relación a las metodologías ágiles:
\begin{itemize}
	\item 2000 horas de experiencia general en trabajo en equipo en proyectos. Un PMP actual o PgMP® va a satisfacer este requisito, pero no está obligado a solicitar el PMI-ACP.
	\item 1500 horas de trabajo en equipos de proyectos específicamente ágiles o con metodologías ágiles. Este requisito se suma a las 2000 horas de experiencia general del proyecto.
	\item 21 horas de formación presencial en prácticas ágiles.
\end{itemize}

En cuanto a la metodología del examen, éste constará de 100 puntos directamente relacionados con las preguntas, y 20 puntos más que se distribuirán por el examen al azar. El temario se organiza en dominios más generales, que a más bajo nivel se componen de tareas concretas. Los temas a tratar, y su puntuación asignada, son los siguientes:
\begin{itemize}
	\item Principios y mentalidad ágiles (16\%): aplicación dentro del equipo y en el contexto del proyecto.
	\item Entrega orientada al valor (20\%): orientación hacia la consecución de objetivos tangibles basados en los \emph{stakeholders} y sus prioridades. Para ello también es necesario que éstos participen desde el inicio del proyecto en la valoración de los objetivos conseguidos.
	\item Participación de las partes interesadas (17\%): promover la participación y colaboración durante todo el ciclo de vida del proyecto de las partes interesadas (\foreign{stakeholders}) presentes y futuras a fin de crear un ambiente de confianza que alinee sus necesidades y expectativas y equilibre sus peticiones con una comprensión del costo / esfuerzo involucrado.
	\item Rendimiento del equipo (16\%): crear un ambiente de confianza en el equipo que promueva la autoorganización y una cultura del alto rendimiento.
	\item Planificación adaptativa (12\%): mantener un plan evolutivo, desde la iniciación hasta el cierre, basado en metas, valores, riesgos, limitaciones, retroalimentación de las partes interesadas y resultados de la revisión.
	\item Detección y resolución del problemas (10\%): basado en la identificación continua de riesgos, en la priorización de éstos, la comunicación de las resoluciones y la prevención para evitar que se repitan.
	\item Mejora continua (9\%): tanto del valor del producto como del proceso y el equipo.
\end{itemize}

Una vez se obtiene la certificación, el acreditado pasa a ser miembro del \emph{Programa de Certificación de Requirimientos (CCR)}. Como tal, tiene la obligación de participar en actividades de desarrollo profesional. La cantidad de actividades a realizar se regula mediante las llamadas \emph{Unidades de Desarrollo Profesional (PDU)}, que son una especie de créditos que se van obteniendo. \emph{PMI-ACP} establece que se necesita conseguir 30 PDUs en un plazo de 3 años para, entonces, optar a un examen de renovación de la certificación.\\

Esta certificación, al ser propiedad de PMI, cuenta con toda infraestructura de apoyo a nivel internacional del Instituto. Así, tiene validez en los 207 países en los que opera PMI, y a nivel local cuenta con los \foreign{Local Chapters} ya mencionados en Madrid y Valencia.\\

Según PMI, es la certificación de su propiedad cuya demanda más ha crecido en los últimos años. Por esto, con una simple búsqueda en Internet se pueden encontrar multitud de cursos de preparación no oficiales a nivel nacional e internacional. Sin embargo, a nivel nacional parece que no tiene mucha demanda, si nos basamos en búsquedas en webs de bolsas de trabajo como \emph{Infojobs}, que no reportan más de 2 o 3 ofertas.


\section{PRINCE2}

PRojects IN Controlled Environments v2 (PRINCE2) es un método basado en procesos para la gestión efectiva de proyectos. Es de dominio público y utilizado en una gran amplitud de campos, desde el campo de las TIC (donde nació en la versión 1 del método) hasta gobiernos como el de Reino Unido.\\

Trata de cubrir, mediante lo que se conoce como \emph{Temáticas}, la Calidad, el Cambio, la estructura de roles del proyecto (Organización), los planes (Cuánto, Cómo, Cuando), el Riesgo y el Progreso del proyecto. El proyecto en cuestión debe ser justificado por un \foreign{Business Case} (o estudio de viabilidad) que debe ser revisado durante el ciclo de vida del proyecto y justificarlo en todo momento como consecución de los beneficios esperados.\\

Se basa en 7 principios:
\begin{enumerate}
	\item Justificación comercial continua
	\begin{itemize}
		\item Se asegura de que hay un motivo justificable para iniciar el proyecto.
		\item La justificación se mantiene válida durante toda la vida del proyecto.
		\item Dicha justificación ha sido identificada, y aprobada.
	\end{itemize}
	\item Aprender de la experiencia
	\begin{itemize}
		\item Se recogen las experiencias anteriores, las que se van obteniendo a lo largo de la ejecución del proyecto, así como las lecciones aprendidas al cierre del mismo.
	\end{itemize}
	\item Roles y Responsabilidades definidos
	\begin{itemize}
		\item Asegurando que los intereses de los usuarios que van a usar el proyecto, los proveedores y el responsable del área de negocio están representados en la toma de decisiones.
	\end{itemize}
	\item Gestión por Fases
	\begin{itemize}
		\item Un proyecto que sigue la metodología PRINCE2 se planifica, se supervisa y se controla fase a fase.
	\end{itemize}
	\item Gestión por excepción
	\begin{itemize}
		\item Es decir, delegar la autoridad suficiente de un nivel de gestión al siguiente, dándole autonomía según unas tolerancias pautadas (de tiempo, coste, calidad, alcance, beneficio y/o riesgo) de manera que, de sobrepasar la tolerancia, se consulte al nivel superior como actuar.
	\end{itemize}
	\item Orientación a productos
	\begin{itemize}
		\item Centra la atención en la definición y entrega de productos, es decir, un proyecto no son un conjunto de tareas a realizar, sino que entrega productos (que se elaboran tras la ejecución de las tareas que sean necesarias).
	\end{itemize}
	\item Adaptación
	\begin{itemize}
		\item Asegurando que la metodología PRINCE2 y los controles a aplicar se basen en el tamaño, complejidad, importancia, capacidad y nivel de riesgo del proyecto.
	\end{itemize}	
\end{enumerate}


Como se ha dicho, aunque la metodología sea pública, desde la organización se ofrecen certificaciones para distintos dominios de dicha metodología. Estas certificaciones están reconocidas a nivel mundial como unas de las más populares en la gestión de proyectos. Se pasan a describir en las siguientes subsecciones.


\subsection{PRINCE2 Foundation}
El certificado \emph{PRINCE2 Foundation} tiene el propósito de confirmar que el usuario tiene suficiente conocimiento y comprensión del método PRINCE2 para poder trabajar eficazmente con, o como miembro de un equipo de gestión de proyectos que trabaje en un entorno que respalde PRINCE2.\\

No tiene requisitos previos y el examen consta de 75 preguntas tipo test de las cuales se deben superar 35 como mínimo y tiene una duración de 60 minutos.

\subsection{PRINCE2 Practitioner}
El certificado \emph{PRINCE2 Practioner} acredita que se ha logrado una comprensión suficiente de cómo aplicar y adaptar PRINCE2 en una situación de escenario. Requiere estar acreditado de al menos una de las siguientes certificaciones:
\begin{itemize}
	\item PRINCE2 Foundation
	\item Project Management Professional (PMP)
	\item Certified Associate in Project Management (CAPM)
	\item IPMA Level A® (Certified Projects Director)
	\item IPMA Level B® (Certified Senior Project Manager)
	\item IPMA Level C® (Certified Project Manager)
	\item IPMA Level D® (Certified Project Management Associate) 
\end{itemize}

El examen consta de 80 preguntas tipo test de las que se deben superar 44. Tiene una duración de 150 minutos.

\subsection{Practitioner Re-Registration}
Es un examen de renovación que deben pasar los acreditados en \emph{Foundation} y \emph{Practitioner} entre los 3 y 5 años desde que se acreditaron para que su certificación siga teniendo validez.

\subsection{PRINCE2 Professional}
\emph{PRINCE2 Professional} es el siguiente paso para los profesionales PRINCE2 que buscan demostrar su experiencia en el método PRINCE2. Este nivel pondrá a prueba su capacidad para gestionar un proyecto PRINCE2 no complejo en todos los aspectos del ciclo de vida del proyecto. Requiere la acreditación previa \emph{PRINCE2 Practitioner}.\\

No hay examen escrito, sino que la prueba consiste en pasarse dos días y medio en un \emph{Centro de Evaluación Residencial} donde se realizarán actividades de grupo y un estudio de un \emph{caso de proyecto} ficticio.