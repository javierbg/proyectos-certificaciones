\section{PMI}

El Project Management Institute es una organización internacional con sede en Pensilvania (Estados Unidos) que asocia a profesionales relacionados con la \emph{gestión de proyectos}.

La organización está presente en casi 100 países y está estructurada en distintas regiones denominadas \textbf{capítulos}. En España existen 3 capítulos distintos, en Barcelona, en Madrid y en Valencia. Para formar parte de estas estructuras a nivel local, hay que hacerse miembro de PMI. La membresía no tiene requisitos y cuesta 129\$ anuales.\\

Sus trabajos más reconocidos son la autoría del \emph{Project Management Body of Knowledge} (PMBOK), el cual es considerado uno de los textos más importantes sobre la gestión de proyectos, y las certificaciones que ofrece, las cuales se detallan a continuación.

\subsection{PMP}

El certificado \emph{Project Management Professional} (PMP) es una de las certificaciones mejor reconocidas en el ámbito de gestión de proyectos.

Para poder obtener esta certificación deben cumplirse los siguientes \emph{requisitos}:

\begin{itemize}
\item 35 horas de formación específica en gestión de proyectos.
\item Disponer de un número mínimo de horas demostrables dedicadas a la gestión de proyectos:
	\begin{itemize}
	\item 4500 horas en 3 años, en el caso de que se tenga formación universitaria.
	\item 7500 horas en 5 años si sólo se tienen estudios de bachillerato.
	\end{itemize} 
\end{itemize}

Cumplir los anteriores requsitos permite presentarse a un \emph{examen} que es necesario aprobar para obtener la certificación. Acceder a este examen tiene un coste de 400\$ si se es miembro del PMI o 550\$ en caso contrario. El examen consta de 200 preguntas de opción múltiple, que cubren seis dominios de la gestión de proyectos de igual forma que están recogidos en el PMBOK:

\begin{enumerate}
\item Inicio
\item Planificación
\item Ejecución
\item Seguimiento y control
\item Cierre
\end{enumerate}

El certificado no es vitalicio, por lo que \emph{es necesario renovarlo cada 3 años}. Para renovarlo es necesario obtener 60 PDUs (Professional Development Units) en este mismo periodo. Las distintas formas de obtener PDUs son la investigación, autoría de artículos, dar conferencias o dedicar horas a la gestión de proyectos.

Esta certificación es \emph{bastante demandada}: una búsqueda rápida en Infojobs a día de hoy (octubre de 2016) da 112 resultados distintos de ofertas de trabajo.

\subsection{CAPM}

El certificado \emph{Certified Associate in Project Management} es el ``hermano menor'' del PMP. Está orientado a profesionales de la gestión de proyectos con menos experiencia que la requerida para el PMP.

Esto queda denotado en sus \emph{requisitos}, más permisivos:

\begin{itemize}
\item 23 horas de formación específica en gestión de proyectos.
\item Tener estudios básicos de secundaria.
\item Disponer de 1500 horas demostrables de experiencia como gestor de proyectos.
\end{itemize}

El \emph{examen} cubre el mismo ámbito que el de PMP, aunque es menos exigente. Acceder a este examen tiene un coste de 225\$ si se es miembro del PMI o 300\$ en caso contrario. Consta de 150 preguntas de opción múltiple.

En cuanto a la \emph{demanda}, las principales ofertas de empleo que se encuentran en Internet son relacionadas con proyectos de software y en su mayoría piden CAPM o PMP.
