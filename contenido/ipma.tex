\section{IPMA}

\subsection{IPMA.world}

La Asociación Internacional para la Dirección de Proyectos o IPMA \emph{(por sus siglas en inglés: International Project Management Association)} es una organización con base en Suiza dedicada al desarrollo y promoción de la dirección de proyectos. Está organizada como una federación internacional de más de 55 asociaciones nacionales de dirección y gestión de proyectos.\\

Su actividad principal es la certificación de las competencias en dirección de proyectos. Para ello ha desarrollado un marco de certificación para las habilidades en dirección de proyectos: el ICB \emph{(IPMA Competence Baseline)}, que sirve de base para su programa de certificación en cuatro niveles. La certificación se lleva a cabo a través de cualquiera de las asociaciones nacionales, y es necesario renovarla cada cierto tiempo (el periodo varía dependiendo del nivel de certificación). La certificación abarca competencias técnicas, contextuales y del comportamiento

\subsubsection{Historia}

El origen de la IPMA se remonta a 1964 cuando un grupo internacional de directores de proyectos se reunió para discutir los beneficios del método de la ruta crítica. Entonces se sugirió para el grupo el nombre de INTERNET \emph{(INTERnational NETwork)}.2 En 1965 este grupo de debate fundó en Suiza una asociación, la actual IPMA, bajo el nombre de IMSA \emph{(International Management Systems Association)}. El primer congreso internacional tuvo lugar en 1967 en Viena, con participantes de 30 países diferentes.

\subsubsection{Certificaciones}

La IPMA establece cuatro niveles de competencia en dirección de proyectos, cada uno de los cuales son certificables a través de la correspondiente certificación. Los niveles de certificación ordenados de menor a mayor nivel de competencia son los siguientes:\\

\begin{itemize}
	\item IPMA Nivel A (Certificado para directores de proyectos).
	\item IPMA Nivel B (Certificado para gestores Senior de proyectos).
	\item IPMA Nivel C (Certificado para gestores de proyectos).
	\item IPMA Nivel D (Certificado para proyectos de gestión asociada).
\end{itemize}

\subsubsection{Competencias}

\subsection{aeipro.com}

