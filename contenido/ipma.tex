\section{IPMA}

\subsection{International Project Management Association (IPMA)}

La Asociación Internacional para la Dirección de Proyectos o IPMA \emph{(por sus siglas en inglés: International Project Management Association)} es una organización con base en Suiza dedicada al desarrollo y promoción de la dirección de proyectos. Está organizada como una federación internacional de más de 55 asociaciones nacionales de dirección y gestión de proyectos.\\

Su actividad principal es la certificación de las competencias en dirección de proyectos. Para ello ha desarrollado un marco de certificación para las habilidades en dirección de proyectos: el ICB \emph{(IPMA Competence Baseline)}, que sirve de base para su programa de certificación en cuatro niveles. La certificación se lleva a cabo a través de cualquiera de las asociaciones nacionales, y es necesario renovarla cada cierto tiempo (el periodo varía dependiendo del nivel de certificación). La certificación abarca competencias técnicas, contextuales y del comportamiento.

El origen de la IPMA se remonta a 1964 cuando un grupo internacional de directores de proyectos se reunió para discutir los beneficios del método de la ruta crítica. Entonces se sugirió para el grupo el nombre de INTERNET \emph{(INTERnational NETwork)}. En 1965 este grupo de debate fundó en Suiza una asociación, la actual IPMA, bajo el nombre de IMSA \emph{(International Management Systems Association)}. El primer congreso internacional tuvo lugar en 1967 en Viena, con participantes de 30 países diferentes.

\subsubsection{Certificaciones}

La IPMA establece cuatro niveles de competencia en dirección de proyectos, cada uno de los cuales son certificables a través de la correspondiente certificación. Los niveles de certificación ordenados de menor a mayor nivel de competencia son los siguientes:\\

\begin{itemize}
	\item \emph{IPMA Nivel A} (Certified Projects Director) Será capaz de gestionar carteras o programas. Para obtenerlo debe tener como mínimo cinco años de experiencia en dirección de carteras, dirección de programas o dirección de multiproyectos.
	\item \emph{IPMA Nivel B} (Certified Senior Project Manager) Será capaz de dirigir proyectos complejos. Para obtenerlo debe tener como mínimo cinco años de experiencia en dirección de proyectos.
	\item \emph{IPMA Nivel C} (Certified Project Manager) Será capaz de dirigir proyectos de complejidad limitada o de gestionar un subproyecto de un proyecto complejo en todos los elementos de competencia de la dirección de proyectos. Para obtenerlo debe tener como mínimo tres años de experiencia en dirección de proyectos.
	\item \emph{IPMA Nivel D} (Certified Project Management Associate) Tendrá conocimientos de dirección de proyectos en todos los elementos de competencia. No es obligatoria ninguna experiencia previa para obtenerlo.
\end{itemize}

La incorporación a los programas de certificación es un incentivo para los directores de proyectos, de programas y de carteras, y para los miembros de los equipos de dirección de proyectos, cuyos beneficios son:

\begin{itemize}
	\item Para el personal de dirección de proyectos: obtener un certificado con reconocimiento internacional que dé fe de su competencia en la dirección.
	\item Para los proveedores de servicios de dirección de proyectos: una demostración de la competencia profesional de sus empleados.
	\item Para los clientes: Mayor certeza de que recibirán los servicios más avanzados de un director de proyectos.
\end{itemize}

\subsubsection{Competencias}

El documento mundial de referencia de la certificación de las competencias de los Directores de Proyectos en IPMA es conocido como ICB (Internacional Competence Baseline), que en España se adapta por el OCDP de AEIPRO como NCB (National Competence Baseline, o Bases para la Competencia en Dirección de Proyectos en su versión 3.0 de fecha Junio 2006). El ojo de la competencia representa la integración de todos los elementos de la Dirección de Proyectos como se ve a través de los ojos de los directores cuando evalúan una situación específica.\\

Para que sea profesional, la disciplina de Dirección de Proyectos tiene que contar con unos rigurosos estándares y directrices a la hora de definir el trabajo del personal de dirección. Estos requisitos han sido definidos recopilando, procesando y normalizando las competencias aceptadas y aplicadas en la dirección de proyectos.\\

De acuerdo con el estándar ISO/IEC 17024 ‘Requisitos generales para organismos que expidan certificados personales', se define “competencia” como la capacidad demostrada para aplicar conocimientos o destrezas, y cuando proceda, atributos personales demostrados. El proceso internacional de certificación comprende todas las actividades mediante las cuales un Organismo Certificador establece que una persona cumple los requisitos de competencia especificados.\\

Una competencia es un compendio de conocimiento, actitud personal, destrezas y experiencia relevante, necesario para tener éxito en una determinada función. Para ayudar a los candidatos a medirse y desarrollarse, y para ayudar a los evaluadores a juzgar la competencia de un candidato, la competencia se desglosa en ámbitos de competencia. Los ámbitos son principalmente dimensiones que, reunidas, describen las funciones y que son más o menos interdependientes. Cada ámbito contiene elementos de competencia que cubren los aspectos de competencia más importantes en él.\\

En la tercera versión de la NCB se decidió describir una dirección de proyectos competente en tres ámbitos distintos:\\
\begin{itemize}
	\item El ámbito de la competencia técnica: para describir los elementos de competencia fundamentales para la dirección de proyectos. Este ámbito cubre el contenido de la dirección de proyectos, en ocasiones citado como los elementos básicos. La NCB contiene 20 elementos de competencia técnica.\\
	\item El ámbito de la competencia de comportamiento: para describir los elementos de competencia personal para la dirección de proyectos. Este ámbito cubre las actitudes y destrezas del director de proyecto. La NCB contiene 15 elementos de competencia de comportamiento.\\
	\item El ámbito de la competencia contextual: para describir los elementos de competencia para la dirección de proyectos relacionados con el contexto de un proyecto. Este ámbito cubre la competencia del director de proyecto para relacionarse dentro de una organización funcional (las operaciones de negocio de la organización permanente a que pertenece el proyecto) y la capacidad para funcionar en una organización por proyectos. La NCB contiene 11 elementos de competencia contextual.\\
\end{itemize}

\subsection{AEIPRO (IPMA en España)}

AEIPRO es una asociación sin ánimo de lucro que tiene por objetivo el desarrollo del campo de la Dirección e Ingeniería de Proyectos. Nace en 1992 y desde el año 1999 es la asociación española representante de IPMA (International Project Management Association).\\

Para conseguir el proceso de certificación en España de IPMA(AEIPRO) se han de seguir los siguientes pasos:\\
\begin{enumerate}
	\item El aspirante a conseguir la certificación en dirección de proyectos debe contactar con la secretaría del OCDP (aeipro@dpi.upv.es) indicando su interés por participar en alguna de las convocatorias previstas y el nivel solicitado.\\
	\item La secretaría remitirá la aceptación preliminar y pedirá al aspirante que envíe la "Solicitud de Evaluación" correspondiente al nivel solicitado, en la que se incluyen y explican los modelos a cumplimentar por el aspirante.\\
	\item El aspirante justifica el abono de la tasa y firma la solicitud devolviéndola a la secretaría, así como los impresos de currículum vitae (formato incluido en la solicitud) y autoevaluación debidamente cumplimentados, y los impresos de justificación de experiencia en su caso.\\ \\ \\
	\item Una vez estudiada la documentación, la secretaría devuelve:
	\begin{itemize}
		\item Aceptación o denegación como candidato para iniciar el proceso de certificación.
		\item El comprobante de pago de la devolución (en caso de no aceptación).
		\item Notificación para la asistencia a la prueba.
	\end{itemize}
	\item El candidato realiza la prueba.\\
	\item La secretaría informa de los resultados directamente al candidato y de los pasos siguientes que procedan.\\
	\item La secretaría actualiza los registros de certificados para que los recién incorporados figuren en el Libro de Certificados de IPMA del año en curso. Y, finalmente, se entrega el diploma acreditativo al ahora ya recién certificado.\\
\end{enumerate}
