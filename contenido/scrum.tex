\newpage{}
\section{Scrum y Scrum Master}

Scrum es un marco de trabajo ágil, en el cual se aplican una serie de buenas prácticas para trabajar en equipo, desarrollar y mantener productos complejos. El Scrum Master es la persona conocedora del proceso Scrum, y se encarga de orientar a las personas involucradas en el proyecto (equipo, propietario del producto,...) para que sigan el proceso determinado por Scrum.

Jeff Shutherland y Ken Schwaber concibieron el proceso Scrum a principios de los años 90. En 2002 fundaron la Scrum Allianze para organizar programas de aprendizaje y certificación de Scrum Master (CSM). En 2009, ya había más de 60000 certificaciones CSMs, más organizaciones usando procesos ágiles que en cascada, y de ellos el 84\% utilizaban Scrum.

En 2006 Ken Schwaber abandonó la Scrum Allianze, principalmente debido a que según él, la organización había dejado a un lado el objetivo de mejorar los procesos de desarrollo de software, para centrarse en hacer dinero. Según sus propias palabras, tenía que elegir entre dos opciones, hacer dinero o hacer lo correcto, tomó la segunda opción, y para ello fundo Scrum.org, para continuar mejorando la calidad y eficacia de Scrum.

\subsection{Certificaciones SCRUM Master}

Una persona que logra un certificado Scrum Master aumenta las oportunidades de trabajar en sectores empresariales que están adoptando métodos ágiles. Demuestra a los empleados y compañeros que conoce los aspectos fundamentales de la metodología Scrum.


\subsubsection{Certified Scrum Master (CSM)}

La organización responsable del certificado CSM es la Scrum Allianze. Para tener acceso a esta certificación, es obligatorio realizar un curso además de realizar una evaluación tipo test. El curso es presencial de dos días (16 horas). Una vez terminado el curso, es obligatorio pasar un examen de 60 minutos de duración, que consta de 35 preguntas de tipo verdadero/falso o de múltiple opción, de las cuales hay que acertar un mínimo de 24 para conseguir la certificación. El pago del curso presencial de dos días incluye derecho a realizar dos intentos de examen. Es necesario renovar el certificado cada dos años. La tasa de renovación es de \$200.


\subsubsection{Proffesional Scrum Master (PSM)}
Scrum.org es la organización responsable de las certificaciones PSM. Esta certificación está ampliamente reconocida en la industria. No requiere revalidaciones ni renovaciones. La Scrum.org dispone de tres tipos de evaluaciones para otorgar diferentes niveles de certificación PSM: 

\begin{itemize}

\item{PSM I: Se certifica un conocimiento básico sobre las funciones de un Scrum Master. La evaluación está basada en The Scrum Guide\cite{PSF02} (guía que describe los elementos básicos de Scrum), por lo que la mayoría de las preguntas son relativas a esta guía. Las características de la evaluación son:}
	
	\subitem{Tasa de \$150 por intento}
	\subitem{Superar el 85\% de la evaluación}
	\subitem{Duración: 60 minutos}
	\subitem{80 preguntas}
	\subitem{Formato: múltiple elección, múltiple respuesta o verdadero/falso}
	\subitem{Idioma: Inglés}

Existe una evaluación de prueba que se puede realizar de forma gratuita, denominada Scrum Open.
Para realizar la evaluación, debes comprar una contraseña para poder acceder.

\item PSM II: Se certifica un conocimiento avanzado sobre las funciones de un Scrum Master. Las personas que superan esta evaluación demuestran dominar los principios básicos de Scrum y son capaces de aplicarlos en situaciones complejas del mundo real.  Las características de la evaluación son:

	\subitem{Tasa de \$250 por intento}
	\subitem{Superar el 85\% de la evaluación}
	\subitem{Duración: 90 minutos}
	\subitem{30 preguntas}
	\subitem{Formato: multiple elección, múltiple respuesta o verdadero/falso}
	\subitem{Idioma: Inglés}

El curso disponible para preparar PSM II se denomina 	Proffesional Scrum Master, el cual tampoco es obligatorio para poder acceder a la evaluación. Una vez finalizado el curso, nos proporcionan una contraseña gratuita para la evaluación del PSM I y un 40\% de descuento para la evaluación PSM II.

\item PSM III: Se certifica un conocimiento distinguido sobre las tareas del Scrum Master. Superando esta evaluación se demuestra tener un conocimiento profundo sobre cómo aplicar Scrum en variedad de equipos y situaciones organizativas complejas. Se requiere haber superado las evaluaciones PSM I y PSM II. Esta evaluación incluye preguntas basadas en casos prácticos. Las características de la evaluación son

	\subitem{Tasa de \$500 por intento}
	\subitem{Superar el 85\% de la evaluación}
	\subitem{Duración: 120 minutos}
	\subitem{Formato: multiple elección, casos prácticos}
	\subitem{Idioma: Inglés}

\end{itemize}

\newpage
\section{Comparativa CSM y PSM}

\subsection{Profesores}

Sobre lo que cada organización exige a sus profesores cabe destacar que la Scrum Alliance no requiere a sus profesores tener experiencia en la industria del desarrollo del software. Sin embargo Scrum.org exige que sus profesionales tengan experiencia como desarrolladores software.

%% Scrum.org trainers must have experience as software developer. It means that every Scrum.org trainers can answer questions about spikes, emergent design/architecture, test-first/driven approaches, failing tests as a means to assess progress, fitness-for-purpose vs gold-plating, technical debt, clean code, etc.

\subsection{Cursos}

\begin{table}[htbp]
\begin{center}
\begin{tabular}{|l|l|l|l|l|l|}
\hline
      &  CSM              & PSM         \\
      &  (Scrum Alliance) & (Scrum.org) \\
\hline \hline
Contenido &               &             \\ \hline
Duración  &               &             \\ \hline
Nivel     &               &             \\ \hline


\end{tabular}
\caption{Comparativa cursos CSM PSM}
\label{tabla:comp_cursos}
\end{center}
\end{table}

\subsection{Certificaciones}

\subsubsection{Cómo se consiguen}

\subsubsection{Formato de evaluaciones y dificultad}

\subsubsection{Coste}

\subsubsection{Mantenimiento}

\newpage{}
\section{El PMBOK}

PMBOK es un acrónimo de Guide to The Project Management Body of Knowledge (Guía de fundamentos de gestión de proyectos). Es una publicación perteneciente al Project Management Institute (PMI), en la cual se documentan una serie de normas, recomendaciones y buenas prácticas aplicables al campo de la gestión de proyectos. Provee un marco de referencia formal para desarrollar proyectos. Proporciona a los jefes de proyecto una guía para avanzar en los procesos necesarios para obtener resultados y alcanzar los objetivos. 

La primera edición del PMBOK fue publicada en 1996. Cada edición sustituye a la anterior y agrega nuevos estándares y buenas prácticas. La última versión disponible es la quinta edición, publicada en 2013. La versión 3, publicada en 2009 está reconocida como estándar internacional por la IEEE (IEEE Std 1490-2011).
