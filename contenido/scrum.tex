\newpage{}
\section{Scrum y Scrum Master}

Scrum es un marco de trabajo ágil, en el cual se aplican una serie de buenas prácticas para trabajar en equipo, desarrollar y mantener productos complejos. El Scrum Master es la persona conocedora del proceso Scrum, y se encarga de orientar a las personas involucradas en el proyecto (equipo, propietario del producto,...) para que sigan el proceso determinado por Scrum.

Jeff Shutherland y Ken Schwaber concibieron el proceso Scrum a principios de los años 90. En 2002 fundaron la Scrum Allianze para organizar programas de aprendizaje y certificación de Scrum Master (CSM). En 2009, ya había más de 60000 certificaciones CSMs, más organizaciones usando procesos ágiles que en cascada, y de ellos el 84\% utilizaban Scrum.

En 2006 Ken Schwaber abandonó la Scrum Allianze, principalmente debido a que según él, la organización había dejado a un lado el objetivo de mejorar los procesos de desarrollo de software, para centrarse en hacer dinero. Según sus propias palabras, tenía que elegir entre dos opciones, hacer dinero o hacer lo correcto, tomó la segunda opción, y para ello fundo Scrum.org, para continuar mejorando la calidad y eficacia de Scrum.

\subsection{Certificaciones SCRUM Master}

Una persona que logra un certificado Scrum Master aumenta las oportunidades de trabajar en sectores empresariales que están adoptando métodos ágiles. Demuestra a los empleados y compañeros que conoce los aspectos fundamentales de la metodología Scrum.


\subsubsection{Certified Scrum Master (CSM)}

La organización responsable del certificado CSM es la Scrum Allianze. Para tener acceso a esta certificación, es obligatorio realizar un curso además de realizar una evaluación tipo test. El curso es presencial de dos días (16 horas). Una vez terminado el curso, es obligatorio pasar un examen de 60 minutos de duración, que consta de 35 preguntas de tipo verdadero/falso o de múltiple opción, de las cuales hay que acertar un mínimo de 24 para conseguir la certificación. El pago del curso presencial de dos días incluye derecho a realizar dos intentos de examen. Es necesario renovar el certificado cada dos años. La tasa de renovación es de \$200.


\subsubsection{Proffesional Scrum Master (PSM)}
Scrum.org es la organización responsable de las certificaciones PSM. Esta certificación está ampliamente reconocida en la industria. No requiere revalidaciones ni renovaciones. La Scrum.org dispone de tres tipos de evaluaciones para otorgar diferentes niveles de certificación PSM: 

\begin{itemize}

\item{PSM I: Se certifica un conocimiento básico sobre las funciones de un Scrum Master. La evaluación está basada en The Scrum Guide\cite{PSF02} (guía que describe los elementos básicos de Scrum), por lo que la mayoría de las preguntas son relativas a esta guía. Las características de la evaluación son:}
	
	\subitem{Tasa de \$150 por intento}
	\subitem{Superar el 85\% de la evaluación}
	\subitem{Duración: 60 minutos}
	\subitem{80 preguntas}
	\subitem{Formato: múltiple elección, múltiple respuesta o verdadero/falso}
	\subitem{Idioma: Inglés}

Existe una evaluación de prueba que se puede realizar de forma gratuita, denominada Scrum Open.
Para realizar la evaluación, debes comprar una contraseña para poder acceder.

\item PSM II: Se certifica un conocimiento avanzado sobre las funciones de un Scrum Master. Las personas que superan esta evaluación demuestran dominar los principios básicos de Scrum y son capaces de aplicarlos en situaciones complejas del mundo real.  Las características de la evaluación son:

	\subitem{Tasa de \$250 por intento}
	\subitem{Superar el 85\% de la evaluación}
	\subitem{Duración: 90 minutos}
	\subitem{30 preguntas}
	\subitem{Formato: múltiple elección, múltiple respuesta o verdadero/falso}
	\subitem{Idioma: Inglés}

El curso disponible para preparar PSM II se denomina 	Proffesional Scrum Master, el cual tampoco es obligatorio para poder acceder a la evaluación. Una vez finalizado el curso, nos proporcionan una contraseña gratuita para la evaluación del PSM I y un 40\% de descuento para la evaluación PSM II.

\item PSM III: Se certifica un conocimiento distinguido sobre las tareas del Scrum Master. Superando esta evaluación se demuestra tener un conocimiento profundo sobre cómo aplicar Scrum en variedad de equipos y situaciones organizativas complejas. Se requiere haber superado las evaluaciones PSM I y PSM II. Esta evaluación incluye preguntas basadas en casos prácticos. Las características de la evaluación son

	\subitem{Tasa de \$500 por intento}
	\subitem{Superar el 85\% de la evaluación}
	\subitem{Duración: 120 minutos}
	\subitem{Formato: multiple elección, casos prácticos}
	\subitem{Idioma: Inglés}

\end{itemize}



\newpage{}
\section{La guía del PMBOK}

\subsection{Introducción}

PMBOK es un acrónimo de Guide to The Project Management Body of Knowledge (Guía de fundamentos de gestión de proyectos). Es una publicación perteneciente al Project Management Institute (PMI), en la cual se documentan una serie de normas, recomendaciones y buenas prácticas aplicables al campo de la gestión de proyectos. Provee un marco de referencia formal para desarrollar proyectos. Proporciona a los jefes de proyecto una guía para avanzar en los procesos necesarios para obtener resultados y alcanzar los objetivos. 

La primera edición del PMBOK fue publicada en 1996. Cada edición sustituye a la anterior y agrega nuevos estándares y buenas prácticas. La última versión disponible es la quinta edición, publicada en 2013. La versión 3, publicada en 2009 está reconocida como estándar internacional por la IEEE (IEEE Std 1490-2011).

El propósito de esta guía es por tanto, identificar un conjunto de buenas prácticas que son útiles y aumentan las posibilidades de éxito de la mayor parte de los proyectos. Esto no significa que debamos aplicarlas siempre, es el equipo o la organización los que deben establecer qué es apropiado aplicar en cada proyecto concreto. Además proporciona un vocabulario común y un código de ética y conducta profesional.

Las primeras dos secciones de la guía PMBOK presentan una introducción de los conceptos clave en el ámbito de la dirección de proyectos. A continuación, se van a describir algunos de estos conceptos.

\subsubsection{Definición de proyecto y dirección de proyectos.}

Un proyecto es un esfuerzo temporal que se lleva a cabo para crear un producto, servicio o resultado único. Tiene un principio y un final definidos. Cada proyecto genera un producto servicio o resultado único, tangible o intangible. Debido a la naturaleza única de los proyectos, pueden existir incertidumbres o diferencias en los productos, servicios o resultados que el proyecto genera.

La dirección de proyectos es la aplicación de conocimientos, habilidades, herramientas y técnicas a las actividades del proyecto para cumplir con los requisitos del mismo. Se logra mediante la aplicación e integración adecuadas de una serie de procesos de dirección de proyectos, descritos en la guía, agrupados de manera lógica categorizados en cinco grupos de procesos.

\begin{itemize}
\item{Inicio}
\item{Planificación}
\item{Ejecución}
\item{Monitoreo y Control}
\item{Cierre}
\end{itemize}

\subsubsection{Definición de programa y dirección de programas}

Un programa se define como un grupo de proyectos relacionados, subprogramas y actividades de programas,, cuya gestión se realiza de manera coordinada para obtener beneficios que no se obtendrían si se gestionaran de forma individual.

La dirección de programas consiste en la aplicación de conocimientos, habilidades, herramientas y ténicas a un programa para satisfacer los requisitos del mismo y para lograr unos beneficios y un control que no es posible obtener dirigiendo los proyectos de manera individual. La dirección de programas se centra en las interdependencias entre proyectos y ayuda a determinar el enfoque óptimo para gestionarlas.

\subsubsection{Definición de Portafolios y dirección de portafolios}

Un portafolio consiste en proyectos, programas, subconjuntos de portafolio y operaciones gestionados como un grupo con objeto de alcanzar los objetivos estratégicos. Los proyectos o gprogramas del protafolio no son necesariamente interdependientes ni están necesariamente relacionados de manera directa.

La dirección de portafolios consiste en la gsetión centralizada de uno o más portafolios con objeto de alcanzar los objetivos estratégicos. La dirección de portafolios se centra en asegurar que los proyectos y programas se revisen a fin de establecer prioridades para la asignación de recursos, y en que la dirección del portafolio sea consistente con las estrategias de la organización y esté alineada con ellas.

\subsubsection{Relación entre dirección de portafolios, dirección de programas, dirección de proyectos y dirección organizacional de proyectos}

La dirección de portafolios, la dirección de programas y la dirección de proyectos son impulsadas por las estrategias organizacionales. La dirección de portafolios se alinea con las estrategias organizacionales mediante la selección de los programas o proyectos adecuados, el establecimiento de prioridades con respecto al trabajo a realizar y la provisión de los recursos necesarios, mientras que la dirección de programas adecua sus proyectos y componentes de programas y controla las interdependencias a fin de lograr los beneficios estipulados. La dirección de proyectos desarrolla e implementa planes para lograr un alcance determinado que viene dado por los objetivos del programa o del portafolio al cual está vinculado, y, en último término por las estrategias organizacionales.

Los proyectos incluidos en programas o portafolios constituyen un medio para alcanzar las metas y los objetivos de la organizacion, a menudo en el contexto de un plan estratégico. Si bien dentro de un progrma un grupo de proyectos puede tener beneficios específicos, estos proyectos también pueden contribuir a los beneficios del programa, a los objetivos del portafolio y al plan estratégico de la organización.

Uno de los objetivos de la dirección de portafolios consiste en maximizar el valor del portafolio mediante un examen cuidadoso de sus componentes: los programas, los proyectos y otros trabajos relacionados que lo constituyen. Los componentes cuya contribución a los objetivos estratégicos del portafolio es mínima, pueden ser excluidos. Paralelamente, los proyectos realimentan a los programas y portafolios mediante informes de estado, lecciones aprendidas y solicitudes de cambio que pueden ayudar a identificar posibles impatos sobre otros proyectos, programsa o paortafolios. Las necesidades de los proyectos, incluidas las necesidades de recursos, se recopilan y se comunican nuevamente a nivel del portafolio, lo que a su vez orienta la planificación de la organización.

\subsubsection{Oficina de Dirección de Proyectos (PMO)}

Es una estructura de gestión que estandariza los procesos de gobierno relacionados con el proyecto y hace más fácil compartir recursos, metodologías, herramientas y técnicas. Las responsabilidades de una PMO pueden abarcar desde el suministro de funciones de soporte para la dirección de proyectos hasta la responsabilidad de la propia dirección de uno o más proyectos. Los directores de proyecto y las PMOs persiguen objetivos diferentes y, por tanto, responden a necesidades diferentes, pero los esfuerzos de ambos están alineados con las necesidades estratégicas de la organización.

\begin{itemize}

\item{El director del proyecto se concentra en los objetivos específicos del proyecto, mientras que la PMO 
gestiona los cambios significativos relativos al alcance del programa, que pueden considerarse como 
oportunidades potenciales para alcanzar mejor los objetivos de negocio.}

\item{El  director  del  proyecto  controla  los  recursos  asignados  al  proyecto  a  fin  de  cumplir  mejor  con  
los  objetivos  del  mismo,  mientras  que  la  PMO  optimiza  el  uso  de  los  recursos  de  la  organización  
compartidos entre todos los proyectos.}

\item{El  director  del  proyecto  gestiona  las  restricciones  (alcance,  cronograma,  costo,  calidad,  etc.)  de  
los  proyectos  individuales,  mientras  que  la  PMO  gestiona  las  metodologías,  estándares,  riesgos/
oportunidades globales, métricas e interdependencias entre proyectos a nivel de empresa.}

\end{itemize}

\subsubsection{El director del proyecto}

El director del proyecto es la persona asignada por la organización para liderar al equipo responsable de alcanzar los objetivos que marca el proyecto. Tiene la responsabilidad de satisfacer las necesidades de las tareas, del equipo y las individuales. Es el nexo de unión entre la estrategia y el equipo, por lo que, para una dirección de proyectos eficaz, se requiere que el director del proyecto cuente con las siguientes competencias:

\begin{itemize}

\item{Conocimiento: Se refiere a lo que el director del proyecto sabe sobre la dirección de proyectos.}

\item{Desempeño: Se refiere a lo que el director del proyecto es capaz de hacer o lograr cuando aplica sus 
conocimientos sobre la dirección de proyectos.}

\item{Personal: Se  refiere  a  la  manera  en  que  se  comporta  el  director  del  proyecto  cuando  ejecuta  
el  proyecto  o  actividades  relacionadas  con  el  mismo.  La  eficacia  personal  abarca  actitudes,  
características básicas de la personalidad y liderazgo, lo cual proporciona la capacidad de guiar al 
equipo del proyecto mientras se cumplen los objetivos del proyecto y se equilibran las restricciones 
del mismo.}

\end{itemize}

\newpage{}
\subsection{Influencia de la organización y ciclo de vida del proyecto}

\subsubsection{Estructuras de la organización}

La estructura de la organización es un factor ambiental de la empresa que puede afectar a la disponibilidad de recursos e influir en el modo de dirigir los proyectos.

\begin{itemize}
\item{Organización funcional clásica: consiste en una jerarquía donde cada empleado tiene un superior claramente definido. En el nivel superior los miembros de la plantilla se agrupan por especialidades. Las especialidades pueden subdividirse en unidades funcionales específicas. Cada departamento de una organización funcional realizará el trabajo del proyecto de forma independiente de los demás departamentos. }

\item{Organización orientada a proyectos: En este tipo de organización los miembros del equipo a menudo están ubicados en un mismo lugar. La mayor parte de los recursos de la organización están involucrados en el trabajo de los proyectos y los directores de proyecto tienen bastante independencia y autoridad. Suelen contar con unidades organizacionales denominadas departamentos, sin embargo pueden reportar directamente al director del proyecto o bien prestar servicios de apoyo a varios proyectos.}

\item{Organización matricial: este tipo de organizaciones son una mezcla de características de las organizaciones funcionales y orientadas a proyectos. Pueden ser, débiles, equilibradas o fuertes, según el nivel de poder e influencia entre gerentes funcionales y directores de proyecto. En las débiles, el rol del director del proyecto es más bien el de un coordinador o facilitador. En las equilibradas, se reconoce la necesidad contar con un director de proyecto, pero no tiene autoridad plena sobre el proyecto. Las organizaciones matriciales fuertes están muy cerca de las orientadas a proyectos, con directores de proyecto con dedicación plena al proyecto y autoridad considerable.}

\item{Organización compuesta: presentan todas las estructuras anteriores pero a diferentes niveles.}

\end{itemize}

\subsubsection{Activos de los procesos de la organización}

Los activos son planes, procesos, políticas, procedimientos y bases de conocimiento específicos de la organización. Incluyen cualquier objeto, práctica o conocimiento de alguna o de todas las organizaciones que participan en el proyecto y que pueden usarse para ejecutar o gobernar el proyecto.

Los procesos y procedimientos de la organización incluyen, entre otros
\begin{itemize}
\item{Inicio y planificación: guías para adaptar los procesos y procedimientos estándar de la organización a las necesidades específicas del proyecto, los estándares de la organización, plantillas, etc.}
\item{Ejecución, Monitoreo y Control: procedimientos de control de cambios, de control financiero, para la gestión de incidentes y defectos, para asignal prioridad, aprobar y emitir autorizaciones, para control de riesgos, requisitos de comunicación, guías con instrucciones de trabajo...}
\item{Cierre: guías o requisitos de cierre del proyecto.}
\end{itemize}

Otro activo es la base de conocimiento de la organización, para almacenar y recuperar información. Puede incluir entre otros, bases de conocimiento de la gestión de configuración, bases de datos financieras, lecciones aprendidas, bases de datos de incidentes, defectos, de mediciones de procesos, archivos de proyectos anteriores, etc.

\subsubsection{Interesados del proyecto}

Un  interesado  es  un  individuo,  grupo  u  organización  que  puede  afectar,  verse  afectado,  o  percibirse  a 
sí  mismo  como  afectado  por  una  decisión,  actividad  o  resultado  de  un  proyecto.  Los  interesados  pueden  participar activamente en el proyecto o tener intereses a los que puede afectar positiva o negativamente la ejecución o la terminación del proyecto. Los interesados pueden ejercer influencia sobre el proyecto, los entregables y el equipo del proyecto a fin de lograr un conjunto de resultados que satisfagan los objetivos estratégicos del negocio u otras necesidades.

El equipo del proyecto identifica a los interesados tanto internos como externos, positivos y negativos, ejecutores y asesores, con objeto de determinar los requisitos del proyecto y las expectativas de todas las partes involucradas. El director del proyecto debe gestionar las influencias de los distintos interesados con relación a los requisitos del proyecto para asegurar un resultado exitoso.

Los interesados tienen diferentes niveles de responsabilidad y autoridad cuando participan en un proyecto, y este puede cambiar durante el ciclo de vida del proyecto. Además, algunos interesados puden impedir el éxito del proyecto, por tanto, requieren la atención del director de proyectos a lo largo de todo el ciclo de vida del proyecto.
